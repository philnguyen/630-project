\documentclass[11pt]{article}

\usepackage{amsfonts}
\usepackage{amsmath}
\usepackage{amsthm}
\usepackage[margin=1in]{geometry}
\usepackage{mathpartir}
\usepackage{bussproofs}
\usepackage{booktabs}
\usepackage{listings}
\usepackage{stmaryrd}
\usepackage{syntax}

\begin{document}

\section{Type Definitions}

\begin{grammar}
  <n> ::= 1
    \alt S <n>

  <x> ::= <ident>

  <t> ::= n | x

  <b> ::= <b> \(\vee\) <b>
    \alt <b> \(\wedge\) <b>
    \alt <b> \(\supset\) <b>
    \alt \(\neg\) <b>
    \alt \(\exists\) <x> . <b>
    \alt \(\forall\) <x> . <b>

  <A> ::= \(\epsilon\)
    \alt <A>, <b>

  <s> ::= <A> \(\vdash\) <b>

  <D> ::= \(\epsilon\)
    \alt <s>, <D>
\end{grammar}

Let $f$ be a countable set of primitive arithmetic functions over
$n$. Let $p$ be a countable set of primitive predicates over $b$.  A
\textbf{minimal term} is a single function ($f$) applied to numerals
($n$). A \textbf{minimal formula} is a single predicate ($p$) applied
to numerals. $s$ is a sequent. A \textbf{derivation} ($D$) is a
list of sequents.

\section{Inference Rules}

\begin{prooftree}

\AxiomC{$A_1, A_2 \vdash b$}
\RightLabel{\tiny interchange}
\UnaryInfC{$A_2, A_1 \vdash b$}

\end{prooftree}

\begin{prooftree}

\AxiomC{$A_1, A_2, A_2 \vdash b$}
\RightLabel{\tiny omit}
\UnaryInfC{$A_1, A_2, \vdash b$}

\end{prooftree}

\begin{prooftree}

\AxiomC{$A_1, A_2, A_3 \vdash b$}
\RightLabel{\tiny conjunction}
\UnaryInfC{$A_1, A_2 \wedge A_3 \vdash b$}

\end{prooftree}

\begin{prooftree}

\AxiomC{$A_1 \vdash b_1$}
\AxiomC{$A_2 \vdash b_2$}
\RightLabel{\tiny $\wedge$ introduction}
\BinaryInfC{$A_1, A_2 \vdash b_1 \wedge b_2$}

\end{prooftree}

\begin{prooftree}

\AxiomC{$A \vdash b_1 \wedge b_2$}
\RightLabel{\tiny $\wedge$ elimination (L)}
\UnaryInfC{$A \vdash b_1$}

\end{prooftree}

\begin{prooftree}

\AxiomC{$A \vdash b_1 \wedge b_2$}
\RightLabel{\tiny $\wedge$ elimination (R)}
\UnaryInfC{$A \vdash b_2$}

\end{prooftree}

\begin{prooftree}

\AxiomC{$A \vdash b_2$}
\RightLabel{\tiny $\vee$ introduction (L)}
\UnaryInfC{$A \vdash b_1 \textbar b_2$}

\end{prooftree}

\begin{prooftree}

\AxiomC{$A \vdash b_1$}
\RightLabel{\tiny $\vee$ introduction (R)}
\UnaryInfC{$A \vdash b_1 \textbar b_2$}

\end{prooftree}

\begin{prooftree}

\AxiomC{$A_1 \vdash b_1 \vert b_2$}
\AxiomC{$A_2, b_1 \vdash b_3$}
\AxiomC{$A_3, b_2 \vdash b_3$}
\RightLabel{\tiny $\vee$ elimination}
\TrinaryInfC{$A_1, A_2, A_3 \vdash b_3$}

\end{prooftree}

\begin{prooftree}

\AxiomC{$A \vdash p(a)$}
\AxiomC{$a$ does not occur in $A$}
\RightLabel{\tiny $\forall$ introduction}
\BinaryInfC{$A \vdash \forall x . p(x)$}

\end{prooftree}

\begin{prooftree}

\AxiomC{$A \vdash \forall x . p(x)$}
\RightLabel{\tiny $\forall$ elimination}
\UnaryInfC{$A \vdash p(a)$}

\end{prooftree}

\begin{prooftree}

\AxiomC{$A \vdash p(a)$}
\RightLabel{\tiny $\exists$ introduction}
\UnaryInfC{$A \vdash \exists x . p(x)$}

\end{prooftree}

\begin{prooftree}

\AxiomC{$A_1 \vdash \exists x . p(x)$}
\AxiomC{$p(a), A_2 \vdash b$}
\AxiomC{$a$ does not occur in $A_1, A_2,$ nor $\exists x . p(x)$}
\RightLabel{\tiny $\exists$ introduction}
\TrinaryInfC{$A_1, A_2 \vdash b$}

\end{prooftree}

\begin{prooftree}

\AxiomC{$A_1, A_2 \vdash b$}
\RightLabel{\tiny $\supset$ introduction}
\UnaryInfC{$A \vdash \exists x . p(x)$}

\end{prooftree}

\begin{prooftree}

\AxiomC{$A_1 \vdash b_1$}
\AxiomC{$A_2 \vdash b_1 \supset b_2$}
\RightLabel{\tiny $\supset$ elimination}
\BinaryInfC{$A_1, A_2 \vdash b_2$}

\end{prooftree}

\begin{prooftree}

\AxiomC{$A_1, A_2 \vdash b$}
\AxiomC{$A_1, A_3 \vdash \neg b$}
\RightLabel{\tiny reductio}
\BinaryInfC{$A_2, A_3 \vdash \neg A_1$}

\end{prooftree}

\begin{prooftree}

\AxiomC{$A \vdash \neg \neg b$}
\RightLabel{\tiny double negation}
\UnaryInfC{$A \vdash b$}

\end{prooftree}

\begin{prooftree}

\AxiomC{$A_1 \vdash p(1)$}
\AxiomC{$p(a), A_2 \vdash p(a+1)$}
\AxiomC{$a$ does not occur in $A_1, A_2, p(1),$ and $p(t)$}
\RightLabel{\tiny complete induction}
\TrinaryInfC{$A \vdash p(t)$}

\end{prooftree}

\section{Proof outline}

\begin{enumerate}
  \item Prove $\vee$, $\exists$, and $\supset$ can be eliminated from
    any derivation (P12).
  \item Show stability of the reduction rules (P13).
  \item Add chain rule to replace structural rules, assert new order
    over derivations. (P14.1)
  \item Define $reduction$. (P14.2)
  \item Show that the $lfp(reduction)$ exists and is finite (P15).
  \item Drop mic, exit stage left.
\end{enumerate}


\end{document}
